    
\subsection{Parameters}
\begin{itemize}  
\item $\rho$, the density, is taken to be time-independent and
  constant; however, in a multi-fluid system
  different fluids can have different value of constant density.
\item $\mu$, the dynamic viscosity can vary arbitrarily in
  time and space; it can also be a function of temperature
  (if the energy equation is included) and strain rate
  invariants (if the stress formulation is selected).
\item $\sigma$, the surface-tension coefficient can vary
  arbitrarily in
  time and space; it can also be a function of temperature
  and passive scalars.
\item $\overline{\beta}$, the effective thermal expansion
  coefficient, is
  assumed time-independent and constant.
\item ${\bf f}(t)$, the body force per unit mass term can
  vary with time, space, temperature and passive scalars.
\item $\rho c_{p}$, the volumetric specific heat, can vary
  arbitrarily with time, space and temperature.
\item $\rho L$, the volumetric latent heat of fusion at a front,
  is taken to be time-independent and constant; however,
  different constants can be assigned to different fronts.
\item $k$, the thermal conductivity, can vary with time,
  space and temperature.
\item $q_{vol}$, the volumetric heat generation, can vary with
  time, space and temperature.
\item $h_{c}$, the convection heat transfer coefficient, can vary
  with time, space and temperature.
\item $h_{rad}$, the Stefan-Boltzmann constant/view-factor product,
  can vary with time, space and temperature.
\item $T_{\infty}$, the environmental temperature, can vary
  with time and space.
\item $T_{melt}$, the melting temperature at a front, is taken
  with time and space; however, different melting temperature
  can be assigned to different fronts.
\end{itemize}
  
In the solution of the governing equations together with
the boundary and initial conditions, Nek5000 treats the
above parameters as pure numerical values; their
physical significance depends on the user's choice of units.
The system of units used is arbitrary (MKS, English, CGS,
etc.). However, the system chosen must be used consistently
throughout. For instance, if the equations and geometry
have been non-dimensionalized, the $\mu / \rho$ in the fluid
momentum equation is in fact
the inverse Reynolds number, whereas if the equations are
dimensional, $\mu / \rho$ represents the kinematic viscosity with
dimensions of $length^{2}/time$.