
\subsection{Internal Boundary Conditions}

In the spatial discretization, the entire computational
domain is subdivided into macro-elements, the boundary
segments shared by any two of these macro-elements
in \(\Omega_f\) and \(\Omega_s\) are denoted as internal boundaries.
For fluid flow analysis with a single-fluid system or heat
transfer analysis without change-of-phase, internal
boundary conditions are irrelevant as the corresponding
field variables on these segments are part of the
solution. However, for a multi-fluid system and for
heat transfer analysis with change-of-phase, special
conditions are required at particular internal
boundaries, as described in the following.

For a fluid system composes of multiple immiscible fluids,
the boundary (and hence the identity) of each fluid must
be tracked, and a jump in the normal traction exists
at the fluid-fluid interface if the surface tension
coefficient is nonzero.
For this purpose, the interface between any two fluids
of different identity must be defined as a special type of
internal boundary, namely, a fluid layer;
and the associated surface tension coefficient also
needs to be specified.

In a heat transfer analysis with change-of-phase, Nek5000 assumes
that both phases exist at the start of the solution, and that
all solid-liquid interfaces are specified as special internal
boundaries, namely, the melting fronts.
If the fluid flow problem is considered, i.e., the energy
equation is solved in conjunction with the momentum and
continuity equations, then only
the common boundary between the fluid and the solid
(i.e., all or portion of \(\partial \overline{\Omega}_f'\) in Fig.~\ref{fig:domains})
can be defined as the melting front.
In this case, segments on \(\partial \overline{\Omega}_f'\) that
belong to the dynamic melting/freezing interface need to be
specified by the user.
Nek5000 always assumes that the density of the two phases
are the same (i.e., no Stefan flow); therefore at the melting
front, the boundary condition for the fluid velocity is the
same as that for a stationary wall, that is, all velocity
components are zero.
If no fluid flow is considered, i.e., only the energy equation
is solved, then any internal boundary can be defined as
a melting front.
The temperature boundary condition at the melting front
corresponds to a Dirichlet
condition; that is, the entire segment maintains a constant temperature
equal to the user-specified melting temperature \(T_{melt}\)
throughout the solution.
In addition, the volumetric latent heat of fusion \(\rho L\)
for the two phases,
which is also assumed to be constant, should be specified.
