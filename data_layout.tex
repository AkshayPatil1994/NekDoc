\section{Data Layout}
\label{sec:data_layout}

Nek5000 was designed with two principal performance criteria in mind,
namely, {\em single-node} performance and {\em parallel} performance.

A key precept in obtaining good single node performance was to use,
wherever possible, unit-stride memory addressing, which is realized by
using contiguously declared arrays and then accessing the data in
the correct order.   Data locality is thus central to good serial 
performance.   To ensure that this performance is not compromised
in parallel, the parallel message-passing data model is used, in which
each processor has its own local (private) address space.  Parallel
data, therefore, is laid out just as in the serial case, save that there
are multiple copies of the arrays---one per processor, each containing 
different data.  Unlike the shared memory model, this distributed memory
model makes data locality transparent and thus simplifies the task of
analyzing and optimizing parallel performance.

Some fundamentals of Nek5000's internal data layout are given below.

\begin{enumerate}
\item
Data is laid out as  $u_{ijk}^e = u(i,j,k,e)$ \\

{\tt   i=1,...,nx1   (nx1 = lx1)} \\
{\tt   j=1,...,ny1   (ny1 = lx1)} \\
{\tt   k=1,...,nz1   (nz1 = lx1} or 1, according to ndim=3 or 2) \\

{\tt   e=1,...,nelv}, where {\tt nelv} $\le$ {\tt lelv}, and {\tt lelv} is the upper
                 bound on number of elements, {\em per processor}.


\item
 Fortran data is stored in column major order (opposite of C).

\item
 All data arrays are thus contiguous, even when {\tt nelv} $<$ {\tt lelv}

\item Data accesses are thus primarily unit-stride (see chap.8 of DFM
   for importance of this point), and in particular, all data on
   a given processor can be accessed as, e.g.,


\begin{verbatim}
   do i=1,nx1*ny1*nz1*nelv
      u(i,1,1,1) = vx(i,1,1,1)
   enddo
\end{verbatim}

   which is equivalent but superior (WHY?) to:

\begin{verbatim}
   do e=1,nelv
   do k=1,nz1
   do j=1,ny1
   do i=1,nx1
      u(i,j,k,e) = vx(i,j,k,e)
   enddo
   enddo
   enddo
   enddo
\end{verbatim}


   which is equivalent but vastly superior (WHY?) to:

\begin{verbatim}
   do i=1,nx1
   do j=1,ny1
   do k=1,nz1
   do e=1,nelv
      u(i,j,k,e) = vx(i,j,k,e)
   enddo
   enddo
   enddo
   enddo
\end{verbatim}


\item All data arrays are stored according to the SPMD programming
   model, in which address spaces that are local to each processor
   are private --- not accessible to other processors except through
   interprocessor data-transfer (i.e., message passing).  Thus

\begin{verbatim}
   do i=1,nx1*ny1*nz1*nelv
      u(i,1,1,1) = vx(i,1,1,1)
   enddo
\end{verbatim}

   means different things on different processors and {\tt nelv} may
   differ from one processor to the next.  (By at most 1, WHY ?)


\item For the most part, low-level loops such as above are expressed in
   higher level routines only through subroutine calls, e.g.,:

\begin{verbatim}
   call copy(u,vx,n)
\end{verbatim}

   where {\tt n:=nx1*ny1*nz1*nelv}.   Notable exceptions are in places where
   performance is critical, e.g., in the middle of certain iterative
   solvers.

\end{enumerate}

