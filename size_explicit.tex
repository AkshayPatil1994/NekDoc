

\begin{description}
\item [ldim]: number of spatial dimensions (2 or 3).% - has to match the setting in the .rea file).
\item [lx1, ly1, lz1]: number of (GLL) points in the \(x\), \(y\) and \(z\) directions, respectively, within each element of mesh1 (velocity) which is equal to the (polynomial order\(+\)1) by definition. \(ly1\) is usually the same as \(lx1\) and for 2D cases \(lz1=1\). \\
\footnote{is \(lx1\ne ly1\) supported?}
\footnote{\(lx1\) recomeneded odd for better performance}
\item [lx2, ly2, lz2]: number of (GLL) points in the \(x\), \(y\) and \(z\) directions, respectively, within each element of mesh2 (pressure). Use \(lx2=lx1\) for PN/PN formulation or \(lx2=lx1-2\) for PN/PN-2 formulation.
\item [lx3, ly3, lz3]: number of (GLL) points in the \(x\), \(y\) and \(z\) directions, respectively, within each element of mesh3.\\
\footnote{mesh3 is rarely used}
\item [lxd, lyd, lzd]: number of points for over integration (dealiasing), use three half rule e.g. for \(lx1=8\) use \(lxd=12\).\\

\item [lelx, lely, lelz]: maximum number of elements per rank for global FDM (Fast Diagonalization Method) solver.
\item [ldimt]:  maximum number of T-array fields (temperature + additional scalars).
\item [lp]: maximum number of ranks.
\item [lelg]: maximum (global) number of elements (it is usually set more than the \# of elements existing in the mesh, for making maximum use of memory is can be set to the exact number of mesh elements).\\
\item [lelt]: maximum number of local elements for T-mesh (per rank, \(lelt \geq lelg/np +1 \)).
\item [lelv]: maximum number of local elements for V-mesh (\(lelv = lelt\)).
\item [lpelv,lpelt,lpert ]: Number of elements of the perturbation field, number of perturbation fields
\item [lpx1, lpy1, lpz1]: Number of point in \(x\),\(y\),\(z\) direction of perturbation field within each element of mesh1
\item [lpx2, lpy2, lpz2]: Number of point in \(x\),\(y\),\(z\) direction of perturbation field within each element of mesh2
\item [lbelv, lbelt]: Total Number of elements of the B-field (MHD)
\item [lbx1, lby1, lbz1]: Number of point in \(x\),\(y\),\(z\) direction of B-field within each element of mesh1
\item [lbx2, lby2, lbz2]: Number of point in \(x\),\(y\),\(z\) direction of B-field within each element of mesh2
\item [lx1m, ly1m, lz1m]: when the mesh is a moving type \(lx1m=lx1\), otherwise it is set to 1.
\item [lxz] : LXZ = LX1*LZ1\\
\code{connect1.f:      common /scruz/  snx(lxz) , sny(lxz) , snz(lxz) ,  efc(lxz)}
\item [lorder]: maximum time integration order (2 or 3).
\item [maxobj]: maximum number of objects. {\textcolor{red}{zero if not using objects?}}
\item [maxmbr]: maximum number of members in an object.
\item [lhis]: maximum number of history points a single rank will read in (NP*LHIS \(<\) number of points in hpts.in).
\item [lctmp0] : {\textcolor{red}{NOT IN USE!}} \\
\code{drive1.f:c      COMMON /CTMP0/ DUMMY0(LCTMP0)}
\item [lctmp1] : \\
\code{drive1.f:c      COMMON /CTMP1/ DUMMY1(LCTMP1)}\\
\code{drive2.f:      COMMON /SCRNS/ WORK(LCTMP1)}
\item [lvec] : {\textcolor{red}{NOT IN USE!}}
\item [mxprev]: maximum number of history entries for residual projection (recommended value: 20).
\item [lgmres]: dimension of Krylov subspace in GMRES (recommended value: 40).
\item [lmvec] : {\textcolor{red}{NOT IN USE !}}
\item [lsvec] : {\textcolor{red}{NOT IN USE!}}
\item [lstore] : {\textcolor{red}{NOT IN USE!}}
\item [maxmor] : \(=lelt\)
\item [lzl]: for 2D cases \(lzl=1\) and for 3D cases \(lzl=3\) (computed automatically).
\end{description}

The following parameters are deprecated and were subsequently removed in newer versions.
\begin{description}
\item [LELGEC] : LELGEC = 1 
\item [LXYZ2] : LXYZ2 = 1 
\item [LXZ21] : LXZ21 = 1 
\item [LMAXV] : LMAXV = LX1*LY1*LZ1*LELV 
\item [LMAXT] : LMAXT = LX1*LY1*LZ1*LELT 
\item [LMAXP] : LMAXP = LX1*LY1*LZ1*LELV 
\end{description}

\begin{comment}
This include file defines the main parameters used to allocate static memory. Note: All parameters have to be >= 1 and parameters in the Z-direction (e.g. LZ1) have to be 1 for a 2D case!


    LDIM: Number of spatial dimensions (2 or 3 - has to match the setting in the .rea file) 

    LX1/LY1/LZ1: Number of points in X,Y,Z direction within each element of mesh1 (velocity) 

    LX2/LY2/LZ2: Number of points in X,Y,Z direction within each element of mesh2 (pressure). Use L{X,Y,Z}2 = L{X,Y,Z}1 (for PN/PN formulation) or L{X,Y,Z}2 = L{X,Y,Z}1 -2 (for PN-2/PN-2 formulation) 

    LX3/LY3/LZ3: Number of points in x,y,z direction within each element of mesh3 

    LXD/LYD/LZD: Number of points for overintegration (dealiasing), use three half rule e.g. for LX1=8 use LXD=12 

    LELX/LELY/LELZ: Maximum number of elements per processor for global fdm solver 

    LDIMT: Maximum number of T-array fields (temperature + additional scalars) 

    LP: Maximum number of processors 

    LELT: Maximum number of local elements for T-mesh (>= int(NELGT/NP) + 1 where NELGT<=LELG and NP<=LP) 

    LELV: Maximum number of local elements for V-mesh (LELV == LELT) 

    LELG: Maximum number of elements 

    LORDER: Maximum time integration order (2 or 3) 

    MXPREV: Maximum number of history entries for residual projection (recommended value: 20) 

    LGMRES: Maximum dimension of GMRES Krylov subspace (recommended value: 40) 

    MAXOBJ: Maximum number of objects 

    MAXMBR: Maximum number of object members 

    LHIS: Maximum number of history points a single processor will read in (NP*LHIS < number of points in hpts.in) 


    LBELV, LBELT: Total Number of elements of the B-field (MHD) 

    LBX1,LBY1,LBZ1: Number of point in x,y,z direction of B-field within each element of mesh1 

    LBX2,LBY2,LBZ2: Number of point in x,y,z direction of B-field within each element of mesh2 

    LPELV,LPELT,LPERT: Number of elements of the perturbation filed, number of perturbation fields 

    LPX1,LPY1.LPZ1: Number of point in x,y,z direction of perturbation field within each element of mesh1 

    LPX2,LPY2.LPZ2: Number of point in x,y,z direction of perturbation field within each element of mesh2 
    
    
   
 Re2file
Jump to: navigation, search

The .re2 file is a binary file containing the mesh and boundary data for a simulation. Typically, this information is in ASCII format within the .rea file. However, for simulations with elements in excess of 50,000, it is encouraged to use this format to decrease the overall memory footprint. Then, both the .rea and .re2 files are needed to successfully run a Nek5000 simulation and lacking either will result in an error message at run time.


Any .rea file can be expanded to the .rea and .re2 file format by using reatore2.


One can also initiate a .rea/.re2 simulation within the genbox tool when creating the simulation's mesh. 


 Boundary Definition
Jump to: navigation, search
Contents [hide] 

    1 General Structure
    2 Types of Boundary Conditions
        2.1 Primitive Types
        2.2 User Defined Types

General Structure

TYPE   ELEMENT FACE   PARAM1        PARAM2        PARAM3        PARAM4       PARAM5

Formats used by the reatore2 tool:

<   1,000 elements   (1X, A3, 2I3,     5G14.6) 
< 100,000 elements   (1X, A3, I5,  I1, 5G14.6) 
else                 (1X, A3, I10, I1, 5G14.6) 

Types of Boundary Conditions

The general rule for boundary conditions in the .rea file is that capital letters indicate that the .rea file or the solver will provide the needed details of the boundary condition, whereas lower case letters indicate that the solver should look in the .usr file for the needed parameters. Since there are no supporting tools that will correctly populate the .rea file with the appropriate values, temperature, velocity, and flux boundary conditions are typically lower case and values must be specified in the USERBC subroutine in the *.usr file. In this case PARAMs in the .rea file are dummies.
Primitive Types

<TYPE>  <Description>                      <Needed Parameters>        <Number of needed Parameters>
E       internal (element connectivity)    adjacent element and face  2
P       periodic                           periodic element and face  2 
T       Dirichlet temperature/scalar       value                      1
V       Dirichlet velocity                 u,v,w                      3
O       outflow                            -                          0
W       wall (no slip)                     -                          0                               
F       flux                               flux                       1
SYM     symmetry                           -                          0
A       axisymmetric boundary              -                          0
MS      moving boundary                    -                          0
I       insulated (zero flux) for temperature                         0
ON      Outflow, Normal (need surface to be normal to x,y,or z        0

User Defined Types

<TYPE>  <Description>               
v        user defined Dirichlet velocity
t        user defined Dirichlet temperature
f        user defined flux

Note:

-Anything else will be taken as zero flux
-TYPE E is just a dummy (connectivity is handled internally by the solver)
-When solving with MHD, the B field must have defined boundary conditions provided after the last nfld provided boundary condition.
-The number of boundary conditions provided must match the number of elements(*number of faces per element) to be solved for in the given field. 
 I.E. when nelt != nelv.

Examples:

P      1       1      6.00000       3.00000       0.00000       0.00000       0.00000

Face 1 of element 1 is periodic with element 6, face 3

T       1       3      200.0         0.00000       0.00000       0.00000       0.00000

Face 3 of element 1 has a constant value equal to 200.0

t      1       3      0.00000       0.00000       0.00000       0.00000       0.00000

Face 3 of element 1 will be assigned a value in USERBC 


 Restart
Jump to: navigation, search

Generic template:

 <Number of lines that follow>  PRESOLVE/RESTART OPTIONS  *****
 <filename>  [<field>]

Examples

    No restart: 

            0 PRESOLVE/RESTART OPTIONS  *****

    Restart from my_old_run0.f0001 

            1 PRESOLVE/RESTART OPTIONS  *****
my_old_run0.f0001  

    Restart from my_old_run0.f0001 and read only U, P, and T fields. 

            1 PRESOLVE/RESTART OPTIONS  *****
my_old_run0.f0001  U P T

NOTE

    The new run must have the same elemental topology.
    The solver will reset the step counter again to 1. Make sure to save your old .fld files you are interested in!
    Make sure the format of the restart file is compatible with PAR067 (Parameters)
    If coordinates (GLL points) were dumped in the field file, they will be used instead of the GLL points calculated based on the spectral element skeleton of the MESH section 
    
    
Initial conditions and drive force data

Not used anymore, use the following dummy lines

0          INITIAL CONDITIONS *****
***** DRIVE FORCE DATA ***** BODY FORCE, FLOW, Q
0 Lines follow.


Variable properties

General case:

***** Variable Property Data ***** Overrides Parameter data.
5 Lines follow.
1 PACKETS OF DATA FOLLOW
   0    1    2  GROUP, FIELD, TYPE OF DATA
group=0 (default)
ifld=1  (field number)
type=2  (0: use CONDUCT/RHOCP from the parameters section, 2: user-specified in USERBC)

For every field a separate packet (last 4 lines) must be provided.

If PAR030>0, it suffices to use

***** Variable Property Data ***** Overrides Parameter data.
0 Lines follow. 


History points

Specify grid points to monitor the specified variables in time. Example:

***** HISTORY AND INTEGRAL DATA *****
1    POINTS.  Var, Hcode, I, J, K, IEL
             UV P    H    2  2  1  4

NOTE:

    Make sure the parameter LHIS in SIZE is large enough 


Output specification

Logical flags to specify the variables to dump in the field file

***** OUTPUT FIELD SPECIFICATION *****
 7   Lines follow.
 T     COORDINATES
 T     VELOCITY
 T     PRESSURE
 T     TEMPERATURE
 F     TEMPERATURE GRADIENT
 1     PASSIVE SCALARS
 T  PS  1


Object specification

PAUL: please add details

***** OBJECT SPECIFICATION *****
      0 Surface Objects
      0 Volume  Objects
      0 Edge    Objects
      0 Point   Objects

 Known Error Messages and Solutions

 ld: library not found for -lcrt1.o
 -Recent versions of Xcode do not include the command line tools as default.  You must be sure to install these.  

 ld: symbol(s) not found for architecture x86_64
 When compiling the Nek tools
 -Check that the maketools script has BIGMEM set to 'false' 

 Using Prenek or Postnek
 If the default screen size is too big and you are unable to see the command prompts and the bottom of the screeen:
 Change the default width and height sizes set in xdriver.c in tools/prenek and tools/postnek.
 The variable to change are windowh and windoww.
 After you adjust these, you will need to recompile the tools.

 Maketools
Jump to: navigation, search

    Modify the maketools script found in nek5_svn/trunk/tools/, according to your environment using your favorite editor 

maketools script

    Run maketools on the commandline (usage: maketools [clean|all|tool(s)]) e.g. 

maketools clean
maketools all
maketools genbox genmap prenek reatore2

NOTE: The maketools script will create, if none exists, a USER/bin/ directory to place all executable tools in. This allows the user to use any tool in any directory by simply typing the proper command. However, specifying a path will allow the default to be overridden.

For example, to place all tools in \$PATH:

maketools all \$PATH

 Reatore2
Jump to: navigation, search

The NEK5000 tool, reatore2 allows users to split an ASCII .rea file to an ASCII .rea and a binary .re2 file. The .re2 file contains the mesh and boundary condition data that is normally written in ASCII in the .rea file. For large simulations, this information can be substantial, so storing it in binary lowers the memory footprint for the simulation.
Running reatore2

    Be sure that your nekton tools are up-to-date and compiled.
    At the command prompt type: reatore2 

NOTE-If the executables for the tools were not placed in the bin directory(default), 
include the path to the reatore2 executable

    User is prompted for name of .rea file 

    -Enter the name to the .rea file, excluding the .rea extenstion 

    User is prompted for the new files name 

    -Enter the name for your new files 


Once reatore2 finishes, there will be two new files, one .rea and one .re2, created from the original .rea file. 

\end{comment}    