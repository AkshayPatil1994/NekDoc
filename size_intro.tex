The SIZE file defines the problem size, i.e. spatial points at which the solution is to be evaluated within each element, number of elements per processor etc.
The SIZE file governs the memory allocation for most of the arrays
in Nek5000, with the exception of those required by the C utilities.
The primary parameters of interest in SIZE are: 
\begin{description}
\item{\bf ldim} = 2 or 3.  This must be set to 2 for
two-dimensional or axisymmetric simulations  (the latter
only partially supported) or to 3 for three-dimensional simulations.
\item{\bf lx1} controls the polynomial order of the approximation,
\(N\)={\tt lx1-1}.
\item{\bf lxd} controls the polynomial order of the integration for
convective terms.  Generally, {\tt lxd=3 * lx1/2}.  On some platforms, however,
it is important for memory access performance that {\tt lx1} and {\tt lxd} be even.
\item{\bf lx2} = {\tt lx1} or {\tt lx1-2}.  This determines the formulation for
the Navier-Stokes\index{Navier-Stokes} solver (i.e., the choice between the \(\PP_N\)--\(\PP_N\) 
or \(\PP_N\)--\(\PP_{N-2}\) methods) and the approximation order for the
pressure, {\tt lx2-1}.
\item{\bf lelt} determines the {\em maximum} number of elements 
{\em per processor}.
\end{description}

The total size of the problem is {\tt lx1*ly1*lz1*lelt}.

\subsection{Memory Requirements}

Per-processor memory requirements for  Nek5000 scale
roughly as 400 8-byte words per allocated gridpoint.  The number
of {\em allocated} gridpoints per processor is 
\(n_{\max}\)={\tt lx1*ly1*lz1*lelt}.  
(For 3D, {\tt lz1=ly1=lx1}; for 2D, {\tt lz1=1}, {\tt ly1=lx1}.)  
If required for a particular simulation, more memory may be made
available by using additional processors.  For example, suppose
one needed to run a simulation with 6000 elements of order \(N=9\).
To leading order, the total memory requirements would be
{\tt \(\approx E(N+1)^3\) points \(\times\) 400 (wds/pt) \(\times\) 8 bytes/wd =
6000 \(\times 10^3 \times 400 \times 8 \)= 19.2} GB.  Assuming there
is 400 MB of memory per core available to the user (after accounting
for OS requirements), then one could run this simulation with
{\tt \(P \ge 19,200\) \mbox{MB} \(/( 400 \)\mbox{MB/proc}) \(= 48\)} processors.
To do so, it would be necessary to set {\tt lelt} \(\ge\) 6000/48 = 125.

We note two other parameters of interest in the parallel context:
\begin{description}
\item{\bf lp}, the maximum number of processors that can be used.
\item{\bf lelg}, an upper bound on the number of elements in the simulation.
\end{description}
\noindent
There is a slight memory penalty associated with these variables, so
one generally does not want to have them excessively large.  It is 
common, however, to have lp be as large as anticipated for a given
case so that the executable can be run without recompiling on
any admissible number of processors (\(P_{\mbox{mem}} \le P \le E\),
where \(P_{\mbox{mem}}\) is the value computed above).

