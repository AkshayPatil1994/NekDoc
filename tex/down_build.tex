This chapter provides a quick overview to using Nek5000
for some basic flow problems provided in the {\tt .../examples}
directory.
           
Nek5000 runs under Linux or any Linux-like OS such as MAC, AIX, BG, Cray etc.
The source is maintained in a legacy SVN repository (last updated June 2, 2016)
and up-to-date Git repositories.  Both repositories are available for
download, but we recommended that new users work with the Git repositories.  

\section{The Git Repositories}

\subsection{Downloading the source code}
The Nek5000 source code is hosted on GitHub at \url{https://github.com/Nek5000/Nek5000}.  You can download the source code as a .zip file\footnote{https://github.com/Nek5000/Nek5000/archive/master.zip} or a .tar.gz archive\footnote{https://github.com/Nek5000/Nek5000/archive/master.tar.gz}.  You can also download (or ``clone'') the repository itself.  The following commands will clone the repository in a directory named {\tt \$HOME/Nek5000} (where {\tt \$HOME} denotes your home directory):
\begin{verbatim}
cd && git clone https://github.com/Nek5000/Nek5000.git -b master
\end{verbatim}
You can also clone the repository in any directory of your choice:
\begin{verbatim}
cd $HOME/my-repos/ && git clone https://github.com/Nek5000/Nek5000.git -b master
\end{verbatim}

\subsection{Downloading the example problems}
The Nek5000 example problems are hosted in a second repository on GitHub, \url{https://github.com/Nek5000/NekExamples}.  Likewise, you can download the examples as a .zip archive\footnote{https://github.com/Nek5000/NekExample/archive/master.zip}, a .tar.gz archive\footnote{https://github.com/Nek5000/NekExamples/archive/master.tar.gz}, or you may clone the repository to your home directory:
\begin{verbatim}
cd && git clone https://github.com/Nek5000/NekExamples.git -b master
\end{verbatim}
or a directory of your choice:
\begin{verbatim}
cd $HOME/my-repos/ && git clone https://github.com/Nek5000/Nek5000.git -b master
\end{verbatim}

\subsection{Tools and scripts}

The {\tt Nek5000/tools} directory contains programs for pre- and
post-processing tasks, such as generating meshes from geometry descriptions.  The following commands will build the tools in {\tt \$HOME/bin}:
\begin{verbatim}
cd Nek5000/tools && ./maketools all
\end{verbatim}
By default, the {\tt maketools} script will use gfortran and gcc as the
Fortran and C compilers; you can specify a different set of compilers by
editing {\tt maketools}.  You can also build the tools in another
directory of your choice by providing a second argument to {\tt maketools}; for example:
\begin{verbatim}
cd $HOME/my-repos/Nek5000/tools && ./maketools all $HOME/my-tools-bin/
\end{verbatim}

Besides the compiled tools, there are several convenience scripts in {\tt
Nek5000/bin} that allow you to easily set up and execute runs of Nek5000.
These scripts are executable as-is and do not need to be compiled. 

We recommend that you add the paths to the tools and scripts to the {\tt \$PATH}
variable in your shell's environment.  This will allow you execute the tools
and scripts without typing the full path to the executables.  In the bash
shell, if you have cloned Nek5000 in {\tt \$HOME/Nek5000} and compiled tools in
{\tt \$HOME/bin}, you can edit your executable path with:
\begin{verbatim}
export PATH="$HOME/bin:$HOME/Nek5000/bin:$PATH"
\end{verbatim}
or if the Nek5000 repository and tools are in custom locations, use:
\begin{verbatim}
export PATH="$HOME/my-tools-bin:$HOME/my-repos/Nek5000/bin:$PATH"
\end{verbatim}
In the following examples, we will assume that the tools and scripts have been
added to your {\tt \$PATH}.

\section{The SVN Repository}

\subsection{Downloading the source code and examples}

The SVN repository can be downloaded with the following commands:
\begin{verbatim}
cd && svn co https://svn.mcs.anl.gov/repos/nek5 nek5_svn
\end{verbatim}
This will create a directoy named {\tt nek5\_svn} in your home directory.  The
example problems are included in the same SVN repo and do not need to be downloaded separately.

\subsection{Tools and scripts}
The {\tt nek5\_svn/trunk/tools} directory contains programs for pre- and
post-processing tasks, such as generating meshes from geometry descriptions.  The following commands will build the tools in {\tt \$HOME/bin}:
\begin{verbatim}
cd $HOME/nek5_svn/trunk/tools && ./maketools all
\end{verbatim}
By default, the {\tt maketools} script will use gfortran and gcc as the
Fortran and C compilers; you can specify a different set of compilers by
editing {\tt maketools}.  
%\footnote{In some cases it may be necessary to reduce the memory
%footprint of some of the tools. The procedures are (will be)
%described in \sc{sec. troubleshooting}.}

Besides the compiled tools, there are several convenience scripts in {\tt
nek5\_svn/trunk/tools/scripts} that allow you to easily set up and execute runs
of Nek5000.  These scripts are executable as-is and do not need to be compiled. 

We recommend that you append the paths to the tools and scripts to the {\tt \$PATH}
variable in your shell's environment.  This will allow you execute the tools
and scripts without typing the full path to the executables.  In the bash
shell, you can edit your executable path with:
\begin{verbatim}
export PATH="$HOME/bin:$HOME/nek5_svn/trunk/tools/scripts:$PATH"
\end{verbatim}
In the following examples, we will assume that the tools and scripts have been
added to your {\tt \$PATH}.
