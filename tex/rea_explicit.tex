\subsection{Parameters}
This section tells nek5000
\begin{itemize}
\item If the input file reflects a 2D or 3D job (it should match the \textit{ldim} parameter in the SIZE file).
\item The combination of heat transfer, Stokes, Navier-Stokes, steady or unsteady to be run.
\item The relevant physical parameters.
\item The solution algorithm within nek5000 to use.
\item The timestep size or Courant number to use, or whether to run variable DT (\(dt\)), etc.
\end{itemize}
A .rea file starts with the following three parameters:
\begin{description}
\item [NEKTON VERSION] the version of nek5000
\item [DIMENSIONAL RUN] number of spatial dimensions (NDIM=2,3 - has to match the setting in the SIZE file).
\item [PARAMETERS FOLLOW] the number of parameters which are going to be followed in the .rea file.(NPARAM)
\end{description}
The latter specifies how many lines of .rea file, starting from the next line, are the parameters and have to be read by the program.\\  

% \noindent\Large\textbf{- Available Parameters}\normalsize
\subsection{Available Parameters}
% \begin{center}
%     \begin{tabular}{ | c | c | c | p{10cm} |}
%     \hline
%       number & name & def. value & comments \\ \hline
%       \textbf{P001}   & DENSITY  &  & density for the case of constant properties (see parameter \textbf{P030})\\ \hline
%       \textbf{P002}   & VISCOS   &  & kinematic viscosity (if \(<0 \rightarrow Re\), otherwise \(1/Re\)).\\
%     \hline
%     \end{tabular}
% \end{center}
\begin{description}
\item [P001  DENSITY] density for the case of constant properties (for variable density see parameter \textbf{P030}).
\item [P002  VISCOS]  kinematic viscosity (if \(<0 \rightarrow Re\), otherwise \(1/Re\)).
\item [P003  BETAG] if \(>0\), natural convection is turned on (Boussinesq approximation). {\textcolor{red}{NOT IN USE !}}
\item [P004  GTHETA] model parameter for Boussinesq approximation (see parameter P003). {\textcolor{red}{ NOT IN USE!}}
\item [P005  PGRADX] {\textcolor{red}{ NOT IN USE!}}
\item [P006  ] {\textcolor{red}{ NOT IN USE!}}
\item [P007  RHOCP] navier5.f:      param(7) = param(1)  ! rhoCP   = rho {\textcolor{red}{ NOT IN USE!}}
\item [P008  CONDUCT] conductivity for the case of constant properties (if \(<0\), it defines the Peclet number, see parameter P030). \\
connect2.f:      if(param(8) .lt.0.0) param(8)  = -1.0/param(8)\\
navier5.f:      param(8) = param(2)  ! conduct = dyn. visc
\item [P009  ] {\textcolor{red}{ NOT IN USE!}} (passed to CPFLD(2,3)!)\\
connect2.f:      CPFLD(2,3)=PARAM(9)
\item [P010  FINTIME] if \(>0\), specifies simulation end time. Otherwise, use NSTEP (P011).\\
drive2.f:      FINTIM = PARAM(10)
\item [P011  NSTEP] number of time steps.\\
connect2.f:            param(11) = 1.0\\
drive2.f:      NSTEPS = PARAM(11)
\item [P012  DT] upper bound on time step \(dt\)   (if \(<0\), then \(dt=|P012|\) constant)\\
connect2.f:            param(12) = 1.0\\
drive2.f:      DT     = abs(PARAM(12))
\item [P013  IOCOMM] frequency of iteration histories\\
drive2.f:      IOCOMM = PARAM(13)
\item [P014  IOTIME] if \(>0\), time interval to dump the fld file. Otherwise, use IOSTEP (P015).\\
drive2.f:      TIMEIO = PARAM(14)
\item [P015  IOSTEP] dump frequency, number of time steps between dumps.\\
drive2.f:      IOSTEP = PARAM(15)\\
navier5.f:      if  (iastep.eq.0) iastep=param(15)   ! same as iostep
\item [P016  PSSOLVER] heat/passive scalar solver:
	\subitem 1: Helmholz
	\subitem 2: CVODE
	\subitem 3: CVODE with user-supplied Jacobian
	\subitem Note: a negative number will set source terms to 0.
\item [P017  AXIS]  {\textcolor{red}{ NOT IN USE!}}
\item [P018  GRID] {\textcolor{red}{ NOT IN USE!}}
\item [P019  INTYPE] {\textcolor{red}{ NOT IN USE!}}\\
 connect2.f:            param(19) = 0.0
\item [P020  NORDER]  {\textcolor{red}{ NOT IN USE!}}
\item [P021  DIVERGENCE] tolerance for the pressure solver.\\
drive2.f:      TOLPDF = abs(PARAM(21))\\
hmholtz.f:      if (name.eq.'PRES'.and.param(21).ne.0) tol=abs(param(21))
\item [P022  HELMHOLTZ] tolerance for the velocity solver.\\
drive2.f:      TOLHDF = abs(PARAM(22))\\
hmholtz.f:      if (param(22).ne.0) tol=abs(param(22))\\
hmholtz.f:         if (param(22).lt.0) tol=abs(param(22))*rbn0\\
navier4.f:      if (param(22).ne.0) tol = abs(param(22))
\item [P023  NPSCAL] number of passive scalars.\\
connect2.f:      NPSCAL=INT(PARAM(23))
\item [P024  TOLREL] relative tolerance for the passive scalar solver (CVODE).\\
drive2.f:      TOLREL = abs(PARAM(24))
\item [P025  TOLABS] absolute tolerance for the passive scalar solver (CVODE).\\
drive2.f:      TOLABS = abs(PARAM(25))
\item [P026  COURANT] maximum Courant number (number of RK4 substeps if OIFS is used).\\
drive2.f:      CTARG  = PARAM(26)
\item [P027  TORDER] temporal discretization order (2 or 3).\\
drive2.f:      NBDINP = PARAM(27)
\item [P028  NABMSH] Order of temporal integration for mesh velocity.if 1, 2, or 3 use Adams-Bashforth of corresponding order. Otherwise, extrapolation of order TORDER (P027).\\
\item [P029  MHD\_VISCOS] if \(>0 \rightarrow\) magnetic viscosity, if \(<0 \rightarrow\) magnetic Reynolds number.\\
connect2.f:      if(param(29).lt.0.0) param(29) = -1.0/param(29)\\
connect2.f:      if (param(29).ne.0.) ifmhd  = .true.\\
connect2.f:         cpfld(ifldmhd,1) = param(29)  ! magnetic viscosity
\item [P030  USERVP] if
	\subitem 0: constant properties
	\subitem 1: user-defined properties via USERVP subroutine (each scalar separately)
	\subitem 2: user-defined properties via USERVP subroutine (all scalars at once)
\item [P031  NPERT]  if \(\ne 0\), number of perturbation modes in linearized N-S.\\
connect2.f:      if (param(31).ne.0.) ifpert = .true.\\
connect2.f:      if (param(31).lt.0.) ifbase = .false.   ! don't time adv base flow\\
connect2.f:      npert = abs(param(31)) 
\item [P032  NBCRE2] if \(>0\), number of BCs in .re2 file, 0: all.\\
connect2.f:      if (param(32).gt.0) nfldt = ibc + param(32)-1
\item [P033  ] {\textcolor{red}{ NOT IN USE!}}
\item [P034  ] {\textcolor{red}{ NOT IN USE!}}
\item [P035  ] {\textcolor{red}{ NOT IN USE!}}
\item [P036  XMAGNET] {\textcolor{red}{ NOT IN USE!}}
\item [P037  NGRIDS] {\textcolor{red}{ NOT IN USE!}}
\item [P038  NORDER2] {\textcolor{red}{ NOT IN USE!}}
\item [P039  NORDER3] {\textcolor{red}{ NOT IN USE!}}
\item [P040  ] {\textcolor{red}{ NOT IN USE!}}
\item [P041  ] 1 \(\rightarrow\) multiplicative SEMG\\
hsmg.f:c     if (param(41).eq.1) ifhybrid = .true. \(\leftarrow\) {\textcolor{red}{ NOT IN USE!}}
\item [P042  ] linear solver for the pressure equation, 0 \(\rightarrow\) GMRES or 1 \(\rightarrow\) PCG
\item [P043  ] 0: additive multilevel scheme - 1: original two level scheme.\\
navier6.f:      if (lx1.eq.2) param(43)=1.\\   
navier6.f:            if (param(43).eq.0) call hsmg\_setup         
\item [P044  ] 0=E-based additive Schwarz for PnPn-2; 1=A-based.

\item [P045  ] Free-surface stability control (defaults to 1.0)\\
subs1.f:      FACTOR = PARAM(45)
\item [P046  ] if \(>0\), do not set Initial Condition (no call to subroutine SETICS).\\
drive2.f:      irst = param(46)\\
ic.f:      irst = param(46)        ! for lee's restart (rarely used)\\
subs1.f:      irst = param(46)
\item [P047  ] parameter for moving mesh (Poisson ratio for mesh elasticity solve (default 0.4)).\\
mvmesh.f:      VNU    = param(47)
\item [P048  ] {\textcolor{red}{ NOT IN USE!}}
\item [P049  ] if \(<0\), mixing length factor {\textcolor{red}{ NOT IN USE!}}.\\
drive2.f:c     IF (PARAM(49) .LE. 0.0) PARAM(49) = TLFAC\\
turb.f:      TLFAC = PARAM(49)
\item [P050  ] {\textcolor{red}{ NOT IN USE!}}
\item [P051  ] {\textcolor{red}{ NOT IN USE!}}
\item [P052  HISTEP] if \(>1\), history points dump frequency (in number of steps).\\
prepost.f:      if (param(52).ge.1) iohis=param(52)
\item [P053  ] {\textcolor{red}{ NOT IN USE!}}
\item [P054  ] direction of fixed mass flowrate (1: \(x\)-, 2: \(y\)-, 3: \(z\)-direction). If 0: \(x\)-direction.\\
drive2.f:      if (param(54).ne.0) icvflow = abs(param(54))\\
drive2.f:      if (param(54).lt.0) iavflow = 1 ! mean velocity
\item [P055  ] volumetric flow rate for periodic case;  if p54\(<0\), then p55:=mean velocity.\\
drive2.f:      flowrate = param(55)
\item [P056  ] {\textcolor{red}{ NOT IN USE!}}
\item [P057  ] {\textcolor{red}{ NOT IN USE!}}
\item [P058  ] {\textcolor{red}{ NOT IN USE!}}
\item [P059  ] if \(\neq0\), deformed elements (only relevant for FDM). !=0 \(\rightarrow\) full Jac. eval. for each el.
\item [P060  ] if \(\neq0\), initialize velocity to 1e-10 (for steady Stokes problem).
\item [P061  ] {\textcolor{red}{ NOT IN USE!}}
\item [P062  ] if \(>0\), swap bytes for output.
\item [P063  WDSIZO] real output wordsize (8: 8-byte reals, else 4-byte).\\
prepost.f:      if (param(63).gt.0) wdsizo = 8         ! 64-bit .fld file
\item [P064  ] if \(=1\), restart perturbation solution\\
pertsupport.f:      if(param(64).ne.1) then !fresh start, param(64) is restart flag
\item [P065  ] number of I\/O nodes (if \(< 0\) write in separate subdirectories).
\item [P066  ] Output format: (only postx uses .rea value; other nondefault should be set in usrdat) (if \(\geq 0\) binary else ASCII).\\
connect2.f:         param(66) = 6        ! binary is default\\
connect2.f:         param(66) = 0        ! ASCII
\item [P067  ] read format (if \(\geq 0\) binary else ASCII).
\item [P068  ] averaging frequency in avg\_all (0: every timestep).
\item [P069  ] {\textcolor{red}{ NOT IN USE!}}
\item [P070  ] {\textcolor{red}{ NOT IN USE!}}
\item [P071  ] {\textcolor{red}{ NOT IN USE!}}
\item [P072  ] {\textcolor{red}{ NOT IN USE!}}
\item [P073  ] {\textcolor{red}{ NOT IN USE!}}
\item [P074  ] if \( > 0\) print Helmholtz solver iterations.\\
hmholtz.f:         if (nid.eq.0.and.ifprint.and.param(74).ne.0) ifprinthmh=.true.
\item [P075  ] {\textcolor{red}{ NOT IN USE!}}
\item [P076  ] {\textcolor{red}{ NOT IN USE!}}
\item [P077  ] {\textcolor{red}{ NOT IN USE!}}
\item [P078  ] {\textcolor{red}{ NOT IN USE!}}
\item [P079  ] {\textcolor{red}{ NOT IN USE!}}
\item [P080  ] {\textcolor{red}{ NOT IN USE!}}
\item [P081  ] {\textcolor{red}{ NOT IN USE!}}
\item [P082  ] coarse-grid dimension (2: linear). {\textcolor{red}{ NOT IN USE!}}
\item [P083  ] {\textcolor{red}{ NOT IN USE!}}
\item [P084  ] if \(<0\), force initial time step to this value.

\item [P085  ] set \(dt\) in \textit{setdt}.\\
subs1.f:            dt=dtopf*param(85)
\item [P086  ] {\textcolor{red}{RESERVED !}} if \(\neq0\), use skew-symmetric form, else convective form.\\
drive2.f:      PARAM(86) = 0 ! No skew-symm. convection for now\\
navier1.f:      if (param(86).ne.0.0) then  ! skew-symmetric form
\item [P087  ] {\textcolor{red}{ NOT IN USE!}}
\item [P088  ] {\textcolor{red}{ NOT IN USE!}}
\item [P089  ] {\textcolor{red}{RESERVED !}}
\item [P090  ] {\textcolor{red}{ NOT IN USE!}}
\item [P091  ] {\textcolor{red}{ NOT IN USE!}}
\item [P092  ] {\textcolor{red}{ NOT IN USE!}}
\item [P093  ] number of previous solutions to use for residual projection.\\
(adjust MXPREF in SIZEu accordingly)
\item [P094  ] if \(>0\), start projecting velocity and passive scalars after P094 steps
\item [P095  ] if \(>0\), start projecting pressure after P095 steps
\item [P096  ] {\textcolor{red}{ NOT IN USE!}}
\item [P097  ] {\textcolor{red}{ NOT IN USE!}}
\item [P098  ] {\textcolor{red}{ NOT IN USE!}}
\item [P099  ] dealiasing: 
	\subitem \(<0\):  disable
	\subitem 3:  old dealiasing
	\subitem 4:  new dealiasing
\item [P100  ] {\textcolor{red}{RESERVED !}} pressure preconditioner when using CG solver (0: Jacobi, \(>0\): two-level Schwarz) {\textcolor{red}{or wiseversa?}}
\item [P101  ] number of additional modes to filter (0: only last mode)\\
navier5.f:         ncut = param(101)+1
\item [P102  ] {\textcolor{red}{ NOT IN USE!}}
\item [P103  ] filter weight for last mode (\(<0\): disabled)
\item [P107  ] if \(\neq0\), add it to h2 in sethlm
\item [P116 NELX] number of elements in x for Fast Tensor Product FTP solver (0: do not use FTP).\\
NOTE: box geometries, constant properties only!
\item [P117  NELY] number of elements in y for FTP
\item [P118  NELZ] number of elements in z for FTP
\end{description}
\bigskip


\subsection{Available Logical Switches}
This part of .rea file starts with such a line:
\begin{verbatim}
n   LOGICAL SWITCHES FOLLOW 
\end{verbatim}
where \(n\) is the number of logical switches which is set in the following lines.
\subsection{Logical switches}\label{sec:switches}
Note that by default all logical switches are set to false.
\begin{description}
 \item[IFFLOW] solve for fluid (velocity, pressure).
 \item[IFHEAT] solve for heat (temperature and/or scalars).
 \item[IFTRAN] solve transient equations (otherwise, solve the steady Stokes flow).
 \item[IFADVC] specify the fields with convection.
 \item[IFTMSH] specify the field(s) defined on T mesh  (first field is the ALE mesh).
 \item[IFAXIS] axisymmetric formulation.
 \item[IFSTRS] use stress formulation in the incompressible case.
 \item[IFLOMACH] use low Mach number compressible flow.
 \item[IFMGRID] moving grid (for free surface flow).
 \item[IFMVBD] moving boundary (for free surface flow).
 \item[IFCHAR] use characteristics for convection operator.
 \item[IFSYNC] use mpi barriers to provide better timing information.
 \item[IFUSERVP] user-defined properties (e.g., \(\mu\), \(\rho\)) varying with space and time.
\end{description}

\begin{comment}
 WRITE (6,*) 'IFTRAN   =',IFTRAN
00130          WRITE (6,*) 'IFFLOW   =',IFFLOW
00131          WRITE (6,*) 'IFHEAT   =',IFHEAT
00132          WRITE (6,*) 'IFSPLIT  =',IFSPLIT
00133          WRITE (6,*) 'IFLOMACH =',IFLOMACH
00134          WRITE (6,*) 'IFUSERVP =',IFUSERVP
00135          WRITE (6,*) 'IFUSERMV =',IFUSERMV
00136          WRITE (6,*) 'IFSTRS   =',IFSTRS
00137          WRITE (6,*) 'IFCHAR   =',IFCHAR
00138          WRITE (6,*) 'IFCYCLIC =',IFCYCLIC
00139          WRITE (6,*) 'IFAXIS   =',IFAXIS
00140          WRITE (6,*) 'IFMVBD   =',IFMVBD
00141          WRITE (6,*) 'IFMELT   =',IFMELT
00142          WRITE (6,*) 'IFMODEL  =',IFMODEL
00143          WRITE (6,*) 'IFKEPS   =',IFKEPS
00144          WRITE (6,*) 'IFMOAB   =',IFMOAB
00145          WRITE (6,*) 'IFNEKNEK =',IFNEKNEK
00146          WRITE (6,*) 'IFSYNC   =',IFSYNC
00147          WRITE (6,*) '  '
00148          WRITE (6,*) 'IFVCOR   =',IFVCOR
00149          WRITE (6,*) 'IFINTQ   =',IFINTQ
00150          WRITE (6,*) 'IFCWUZ   =',IFCWUZ
00151          WRITE (6,*) 'IFSWALL  =',IFSWALL
00152          WRITE (6,*) 'IFGEOM   =',IFGEOM
00153          WRITE (6,*) 'IFSURT   =',IFSURT
00154          WRITE (6,*) 'IFWCNO   =',IFWCNO
00155          DO 500 IFIELD=1,NFIELD
00156             WRITE (6,*) '  '
00157             WRITE (6,*) 'IFTMSH for field',IFIELD,'   = ',IFTMSH(IFIELD)
00158             WRITE (6,*) 'IFADVC for field',IFIELD,'   = ',IFADVC(IFIELD)
00159             WRITE (6,*) 'IFNONL for field',IFIELD,'   = ',IFNONL(IFIELD)

\end{comment}